%!TEX TS-program = xelatex
%!TEX encoding = UTF-8 Unicode

\documentclass[12pt, a4paper]{scrartcl}

%% Page Layout
\usepackage[margin=1in]{geometry}

\usepackage{euler} % math font package needs to be loaded before others
\usepackage{xunicode,xltxtra, polyglossia}
\setdefaultlanguage[variant=american]{english}

%% fonts, symbols, text
	%%% fonts
	\usepackage{fontspec} %(include if mathspec is not loaded)
	\defaultfontfeatures{Mapping=tex-text, Ligatures=TeX}
	%%% text decoration
	\usepackage[normalem]{ulem} % sout
	%%% semantics symbols
	\usepackage{stmaryrd}
	\usepackage{amsmath,amssymb}
	\newcommand{\transl}{\rightsquigarrow \ensuremath}
	%%% other symbols
	\usepackage{pifont}% http://ctan.org/pkg/pifont
	\newcommand{\cmark}{\ding{51}}%
	\newcommand{\xmark}{\ding{55}}%

%% layout
%%% page layout
\usepackage{multicol}


%% bibliography
\usepackage[round]{natbib}
\newcommand{\posscite}[1]{\citeauthor{#1}'s (\citeyear{#1})}

%% figures, examples, diagrams
%%% examples
\usepackage{linguex}
\renewcommand{\firstrefdash}{}
%%% tables 
\usepackage{booktabs}
%%% figures
\usepackage{graphics}


%% decoration and features
%%% colors
\usepackage[dvipsnames]{xcolor}

%% bibliography
\renewcommand*{\refname}{\normalsize\textbf{References}\\ \vspace{-.5\baselineskip}}

%% fonts
\setmainfont[Scale=MatchLowercase,Mapping=tex-text, SmallCapsFeatures={Letters=SmallCaps}]{Times New Roman}
\setsansfont[Scale=MatchLowercase,Mapping=tex-text]{Times New Roman}

\usepackage{setspace}
\begin{document}

\bibliographystyle{plainnat}
% \enablehyphenation
% \vspace{-2em}
% \maketitle
\textcolor{white}{.} \vspace{-3.9\baselineskip} \\
\begin{center}
	\textbf{\large%\thetitle
		A diverse family (of sentences)\\ Projectivity differs across embedding operators---but not like you think}
\end{center}


\vspace{-.4\baselineskip}
	\noindent We present experimental evidence that the projectivity of attitude complements varies across different entailment-cancelling operators, and that this effect of entailment-cancelling operator differs between various attitude verb triggers. We find that verbs show patterns of between-operator projection behavior which group in interesting ways, but 
	do not support \posscite{karttunen_observations_1971} long-standing classification of factive vs. semi-factive verbs, which has been assumed throughout the literature \citep[e.g.][]{djarv_cognitive_2018} + OTHERS.

\vspace{-\baselineskip}
\paragraph{Projection across entailment-cancelling operators.}  \hspace{-1em}
	Certain attitude ascriptions come with an inference to the truth of their complement, even if embedded under entailment-cancelling operators (shown for \emph{\lq discover\rq} in \ref{ex:family}), in which case the inference is said to \emph{project} (e.g. \citealp{karttunen_observations_1971}).

	\vspace{-.3\baselineskip}
	\ex. \label{ex:family}
		\a. \label{ex:mod}
			Modals: \hfill
			\emph{\lq Perhaps Cole discovered that Julian dances Salsa.\rq}
		\b. \label{ex:neg}
			Negation: \hfill
			\emph{\lq Cole didn't discover that Julian dances Salsa.\rq}
		\b. \label{ex:q}
			Polar Questions: \hfill
			\emph{\lq Did Cole discover that Julian dances Salsa?\rq}
		\b. \label{ex:cond}
			Conditionals: \hfill
			\emph{\lq If Cole discovered that Julian dances Salsa, Logan will be joyful.\rq}
		\z.
	\z.

	\vspace{-.4\baselineskip}
	% Previous work on projection showed that it is not a categorial property of lexical triggers \citep{tonhauser_how_2018}, but a gradient one, affected by various contextual factors \citep{simons_what_2010,de_marneffe_did_2012,beaver_questions_2017,degen_prior_2021}. In light of this, we expect that the hetergeneous entailment-cancelling operators in \ref{ex:family} affect projection differentially.
	%
	\citet{karttunen_observations_1971} suggested that the entailment-cancelling operators in \ref{ex:family} affect projection differentially and proposed a generalization distinguishing \emph{factive} verbs (\emph{regret, forget, resent}) vs. \emph{semi-factive} verbs (\emph{discover, realize, see, find out, notice}): Factives always project, while semi-factives always project across negation, but not always in polar questions or conditional antecedents. Although widely assumed, experimental evaluations of this distinction have been little. 

	To provide a systematic way of distinguishing these classes, \cite{djarv_cognitive_2018} suggest that they correspond to emotive and cognitive predicates, respectively. In a study assessing the acceptability of affirmative responses to a target utterance while explicitly denying a potentially projective inference \citep[based on][]{cummins_experimental_2012}, they find that this kind of affirmative cancellation is more readily available for emotive than cognitive predicates, suggesting that the main clause content and projective content are logically more independent from each other for emotive predicates. \cite{smith_relationship_2014}, investigating the effect of operator on the projection of various types of projective content, find an effect on ratings of participant surprisal about the inferences in question, suggesting that inferences triggered by \emph{know} and epithets project more from under negation than conditional antecedents whereas non-restrictive relative clauses show the opposite pattern.

	In our work, we analyze measures from a task designed to measure speaker commitment more directly to address the questions: \textbf{(i)} Is projection affected by differences in entailment-canceling environments? \textbf{(ii)} Do these effects vary for different triggers (and in what way)?

\vspace{-\baselineskip}
\paragraph{Experimentally investigating projection,} \hspace{-1em}
	we used a response task to elicit judgments about how strongly a speaker would be committed towards the embedded clause \citep[from][]{tonhauser_prosodic_2016}. We presented sentences like in \ref{ex:family} as asserted by a named speaker (e.g. “\textbf{Daniel:} \emph{\lq Did Cole\dots?\rq}”). Participants then provided a certainty-rating in response to a prompt like: \emph{“Is Daniel certain that Julian dances Salsa?”}, by moving a slider on a scale from \lq no\rq\ (coded as \texttt{0}), to \lq yes\rq\ (coded as \texttt{1}). 

% paragraph experiment (end)

\vspace{-\baselineskip}
\paragraph{Design and Expectations.} \hspace{-1em}
	We compared certainty-ratings for the four entailment-canceling operators in \ref{ex:family}, and 20 clause-embedding predicates (\texttt{verb}: {\em be annoyed, discover, know, reveal, see, acknowledge, admit, announce, confess, confirm, establish, hear, inform, prove, pretend, suggest, say, think, be right, demonstrate}).
	Based on the Karttunen-Djärv generalization about emotive factives vs. cognitive semi-factives, we would expect emotive factives (\emph{be annoyed}), and verbs normally taken to be factive (\emph{know}), to be highly projective in a way that is indifferent to the embedding context, and cognitive semi-factives (\emph{discover, see}) to show higher projectivity ratings under negation compared to questions and conditionals.

% paragraph design_and_expectations_ (end)

\vspace{-\baselineskip}
\paragraph{Method.} \hspace{-1em}
	The study, originally designed to address a different research question, consisted of 12 experiments, all of which manipulated the factor \texttt{verb} for 20 items corresponding to to the content of the complement clause. Experiments 1--3 used polar question embedding, exps. 4--6 used negation, 7--9 modals, and 10--12 conditionals, making the \texttt{operator} manipulation a between-subjects factor. We analyzed data from 2682 self-identified native English speakers participated online across the 12 experiments (recruited via Prolific and Amazon MTurk). Participants saw items Latin-squared and randomized with six control stimuli.

% paragraph method (end)

\vspace{-\baselineskip}
\paragraph{Results and Discussion.} \hspace{-1em}
	Pooling the data across our 12 experiments, we examined the effect on certainty-ratings of \texttt{operator}, and \texttt{verb}, as well as their interaction. Mean certainty-ratings by operator and predicate, and 95\% bootstrapped confidence intervals are shown in  \textbf{Figure \ref{fig:figure1}}.

	% \vspace{-.3\baselineskip}
	% \textbf{version w baseline know/m}
	% \noindent The data was analyzed using a mixed effects linear regression (using \texttt{lme4, lmertest} in \texttt{R}; \citealp{bates_fitting_2015,kuznetsova_lmertest_2016,r_core_team_r_2014}), with \texttt{know} and \texttt{modal} as reference levels, and random intercepts for participants and items. We found highly significant main effects of \texttt{operator} (\texttt{conditional}: $+0.142$, \texttt{negation}: $+0.143$, ; \texttt{question}: $+0.225$, where $p < 2e^{-16}$ in all three cases), as well as many interactions of \texttt{operator} and  \texttt{verb} across the board (where $p < 0.001$ in $38$ cases, $p < 0.01$ in one, and $p < 0.05$ in three out of $57$ possible interactions).\\

	\noindent The data was analyzed using a mixed effects linear regression (using \texttt{lme4, lmertest} in \texttt{R}; \citealp{bates_fitting_2015,kuznetsova_lmertest_2016,r_core_team_r_2014}), with \texttt{be\_annoyed} and \texttt{negation} as reference levels, and random intercepts for participants and items.
	% The mean for this baseline (intercept) is $0.867$.
	We found highly significant main effects of \texttt{operator}: For our baseline \texttt{be\_annoyed}, both \texttt{conditional} and \texttt{modal} are clearly less projective than negation, thereby supporting the claim that the embedding context does matter. We also found many interactions of \texttt{operator} and \texttt{verb} across the board, suggesting that the effect of embedding context diffes by verb. Notably, \emph{\lq discover\rq} is more projective in polar questions and conditional antecedents than under negation, patterning opposite to Karttunen's claims about semi-factives. For the emotive predicate \emph{\lq be annoyed\rq} no significant effect is found for \texttt{question} (vs \texttt{negation}), as would be expected based on Karttunen, but we do find unexpected differences between \texttt{negation} $>$ \texttt{conditional, modal}. \emph{know} shows effects that would be incompatible with a characterization as either factive or semi-factive: $\texttt{question} > \texttt{conditional, negation} > \texttt{modal}$. If \emph{\lq know\rq} is a factive predicate, no difference would be expected. If it is semi-factive, we would, again, expect higher projectivity under negation than in questions and conditionals.

	% for \texttt{conditional} ($-0.116$, $p < 1.6e^{-13}$) and \texttt{modal} ($-0.156$, $p < 2e^{-16}$), while the effect of \texttt{question} was only marginally significant ($+0.025$, $p < 0.1$). We also found many interactions of \texttt{operator} and \texttt{verb} across the board (where $p < 0.001$ in $43$ cases, $p < 0.01$ in three, and $p < 0.05$ in one out of $57$ possible interactions).\\
	
% paragraph results (end)

\vspace{-\baselineskip}
\paragraph{Verb profiles.} % (fold)
	add discussion about verb classes here
% paragraph profiles (end)

\vspace{-\baselineskip}
\paragraph{Conclusion.} % (fold)
	add conclusion here
% paragraph conclusion (end)

\pagebreak
\begin{figure}[t]
		\vspace{-.8\baselineskip}
		\centering
		\includegraphics[width=\textwidth]{graphs/proj-by-both.pdf}\vspace{-1.2\baselineskip}
		\caption{\small Mean certainty ratings by predicate and operator with 95\% bootstrapped confidence intervals (y-axis), by verb (x-axis), and operator (color/grouping, where: \texttt{N} (blue): negation, \texttt{M} (green): modals, \texttt{C} (red): conditional antecedents, \texttt{Q} (purple): polar questions).}
		\label{fig:figure1}
	\end{figure}

\bibliography{projective-content.bib}

\end{document}