\documentclass[11pt,fleqn]{article}
\usepackage[margin=1in,top=1in,bottom=1in]{geometry}
\usepackage{tikz}
\usepackage{mathtools}
\usepackage{longtable}
\usepackage{enumitem}
\usepackage{hyperref}
%\usepackage[dvips]{graphics}
%\usepackage[table]{xcolor}
%\usepackage{amssymb}
\usepackage{float}
%\usepackage{subfig}
\usepackage{booktabs}
\usepackage{subcaption}

\usepackage[normalem]{ulem}

\usepackage{multicol}
\usepackage{txfonts}
\usepackage{amsfonts}
\usepackage{natbib}
\usepackage{gb4e}
\usepackage[all]{xy}
\usepackage{rotating}
\usepackage{tipa}
\usepackage{multirow}
\usepackage{authblk}
\usepackage{url}
\usepackage{pdflscape}
\usepackage{rotating}
\usepackage{adjustbox}
\usepackage{array}


\def\bad{{\leavevmode\llap{*}}}
\def\marginal{{\leavevmode\llap{?}}}
\def\verymarginal{{\leavevmode\llap{??}}}
\def\swmarginal{{\leavevmode\llap{4}}}
\def\infelic{{\leavevmode\llap{\#}}}

\definecolor{airforceblue}{rgb}{0.36, 0.54, 0.66}
%\definecolor{gray}{rgb}{0.36, 0.54, 0.66}

\definecolor{Pink}{RGB}{240,0,120}
\newcommand{\red}[1]{\textcolor{Pink}{#1}}
\newcommand{\jd}[1]{\textbf{\textcolor{Pink}{[jd: #1]}}}

\newcommand{\dashrule}[1][black]{%
  \color{#1}\rule[\dimexpr.5ex-.2pt]{4pt}{.4pt}\xleaders\hbox{\rule{4pt}{0pt}\rule[\dimexpr.5ex-.2pt]{4pt}{.4pt}}\hfill\kern0pt%
}

\setlength{\parindent}{.3in}
\setlength{\parskip}{0ex}

\newcommand{\yi}{\'{\symbol{16}}}
\newcommand{\nasi}{\~{\symbol{16}}}
\newcommand{\hina}{h\nasi na}
\newcommand{\ina}{\nasi na}

\newcommand{\foc}{$_{\mbox{\small F}}$}

\hyphenation{par-ti-ci-pa-tion}

\setlength{\bibhang}{0.5in}
\setlength{\bibsep}{0mm}
\bibpunct[:]{(}{)}{,}{a}{}{,}

\newcommand{\6}{\mbox{$[\hspace*{-.6mm}[$}} 
\newcommand{\9}{\mbox{$]\hspace*{-.6mm}]$}}
\newcommand{\sem}[2]{\6#1\9$^{#2}$}
\renewcommand{\ni}{\~{\i}}

\newcommand{\citepos}[1]{\citeauthor{#1}'s \citeyear{#1}}
\newcommand{\citeposs}[1]{\citeauthor{#1}'s}
\newcommand{\citetpos}[1]{\citeauthor{#1}'s (\citeyear{#1})}

\newcolumntype{R}[2]{%
    >{\adjustbox{angle=#1,lap=\width-(#2)}\bgroup}%
    l%
    <{\egroup}%
}
\newcommand*\rot{\multicolumn{1}{R{90}{0em}}}% no optional argument here, please!


\title{Does at-issueness predict projection? Further investigations of the Gradient Projection Principle}

%\thanks{For helpful comments on the research presented here, we thank David Beaver, Cleo Condoravdi, Kai von Fintel, Lauri Karttunen, Mandy Simons, Greg Scontras, the anonymous reviewers for {\em Semantics and Linguistic Theory} 2018, as well as the audiences at the MIT Linguistics colloquium, the 2018 Annual Meeting of XPRAG.de and at the University of T\"ubingen. We gratefully acknowledge financial support for this research from {\em National Science Foundation} grant BCS-1452674 (JT) and the Targeted Investment for Excellence Initiative at The Ohio State University (JT). IGOR Tuebingen, SALT TALK}}

\author{Author(s)}

%\author[$\circ$]{Judith Tonhauser}
%\author[$\bullet$]{Judith Degen}
%\affil[$\circ$]{The Ohio State University / University of Stuttgart}
%\affil[$\bullet$]{Stanford University}
%
%\renewcommand\Authands{ and }

\newcommand{\jt}[1]{\textbf{\color{blue}JT: #1}}

\begin{document}

%\tableofcontents
%\newpage

\maketitle

\vspace*{-1cm}

\begin{abstract}

\citealt{tbd-variability} hypothesized that at-issueness is one factor that modulates the projection of content (see also \citealt{brst-salt10,brst-ar}). Specifically, according to their Gradient Projection Principle, if content $C$ is expressed by a constituent embedded under an entailment-canceling operator, then $C$ projects to the extent that it is not at-issue. Their work provided evidence for the GPP from xx contents that had been described as projective in the literature, at-issueness was measured in two different ways, only one consistently brought evidence in support (question, asking whether, are you sure? mixed results 2b). This paper provides further investigate the GPP by investigating a) more varied items, b) with more types of embedding, and c) other measures of at-issueness. These experiments provide additional support for the GPP, and they also show: comparison of projection across entailment-canceling operators, comparison of at-issueness diagnostics. Suggests that the diagnostics may not all measure the same (\citealt{snider2017}).

\end{abstract}

		
\section{Introduction}\label{s1}

\begin{exe}
\ex\label{gpp} {\bf Gradient Projection Principle} \hfill (\citealt[400]{tbd-variability}) \\ If content $C$ is expressed by a constituent embedded under an entailment-canceling operator, then $C$ projects to the extent that it is not at-issue.
\end{exe}


\begin{exe}
\ex\label{rqs} {\bf Main research question} \\
Are not-at-issueness and projection positively correlated, as predicted by the Gradient Projection Principle?
\end{exe}

\begin{exe}
\ex\label{rqs} {\bf Ancillary research questions}
\begin{xlist}

\ex Is the at-issueness of contents embedded under different entailment-canceling operators positively correlated?

\ex Is the at-issueness of contents positively correlated under different measures of at-issueness?
\end{xlist}
\end{exe}


Previous investigations of the relationship between projection and at-issueness:

\begin{itemize}

\item \citealt{brst-salt10,brst-ar}

\item \citealt{xue-onea11}

\item \citealt{tbd-variability}

\end{itemize}

Limited how? (only presuppositions + CCs of factive predicates: \citealt{degen-tonhauser-factive} / few measures of at-issueness / some supporting results, some not / some embedded, other matrix clause)

\begin{exe}
\ex\label{pred} 20 clause-embedding predicates 

\begin{xlist}

\ex canonically factive: {\em be annoyed, discover, know, reveal, see}

\ex nonfactive:

\begin{xlist}

\ex nonveridical nonfactive: {\em pretend, suggest, say, think}

\ex veridical nonfactive: {\em be right, demonstrate}

\end{xlist}

\ex optionally factive: {\em acknowledge, admit, announce, confess, confirm, establish, hear, inform, prove}

\end{xlist}

\end{exe}

\section{Defining and diagnosing at-issueness}\label{s2}

\begin{itemize}

\item Todor Koev

\item Natasha Korotkova

\item Murray, Anderbois et al

\item \citealt{snider2017,snider-paper}

\item \citealt{syrett-koev2015}

\item \citealt{tonhauser-sula6}

\item Diagnostics for at-issueness are either suitable for polar questions or for declaratives.

\begin{itemize}

\item Polar questions: at-issue content partitions the context set

`asking whether'

`yes, $c$'

`yes, but not-c''

\item Declaratives: at-issue content is the target of anaphora

`sure that'

`yes, that's true, $c$'

`yes, that's true, but not-c''

\end{itemize}

\end{itemize}

\section{Experiments 1-3: Methods}\label{s3}

Exps.~1-3 were designed to investigate whether not-at-issueness and projection are positively correlated, as predicted by the Gradient Projection Principle, as well as to compare various measures of at-issueness.\footnote{\label{f-github}The experiments, data, and R code for generating the figures and analyses of the experiments reported in this paper are available at \url{https://github.com/judith-tonhauser/projective-probability}. All experiments were conducted with IRB approval. Exps.~1b, 2b, and 3b were preregistered: \url{https://osf.io/cxq47}.}

To measure projection, Exps.~1-3 employed the `certain that' diagnostic for projection, which is suitable to measure projection for both polar questions and declarative sentences (see also, e.g.\ \citealt{tonhauser-salt26,djaerv-bacovcin-salt27,stevens-etal2017,lorson2018,tbd-variability,mahler-nels,mahler2020,demarneffe-etal-sub23}).\footnote{For other diagnostics for projection see, e.g.\ \citealt{smith-hall11,xue-onea11}, and \citealt{brst-lang11}; see also the discussion in \citealt{tbd-variability}.}  On this diagnostic, participants are presented with an utterance, like one of the utterances in (\ref{proj-stim}), in which the content to be diagnosed for projection (here, that Julian dances salsa) occurs embedded under an entailment-canceling operator:

\begin{exe}

\ex\label{proj-stim} `certain that' diagnostic for projection
\begin{xlist}
\ex Daniel: Did Cole discover that Julian dances salsa?
\ex Daniel: Cole didn't discover that Julian dances salsa.
\ex Daniel: Perhaps Cole discovered that Julian dances salsa.
\ex Daniel: If Cole discovered that Julian dances salsa, Logan will be joyful.
\end{xlist}
\end{exe}
To assess projection, participants are asked whether the speaker is certain of the content; for instance, in \ref{proj-stim}, participants are asked whether Daniel is certain that Julian dances salsa. If a participant takes the speaker to be certain of the content, we assume that the content projects; if a participant does not take the speaker to be certain of the content, we assume that the content does not project. Following \citealt{tbd-variability}, the `certain that' diagnostic was implemented in Exps.~1-3 with a gradient response scale: participants gave their responses on a slider marked `no' at one end and `yes' at the other. We assume that the closer to `yes' a participant's response is, the more the speaker is certain of the content, that is, the more projective the content is. In other words, we assume that gradient certainty ratings reflect gradience in the speaker's commitment to the truth of the content.\footnote{Strictly speaking,  gradient certainty ratings reflect gradience in the degree to which participants \emph{perceive} the speaker to be committed to the truth of the content. That is, as in any experiment, the quantity of interest, in this case speaker commitment, is only indirectly measured.} As discussed in \citealt{tbd-variability}, a second interpretation of gradient certainty ratings is that they reflect a participant's degree of belief in the speaker's (binary) commitment to the truth of the content. While we adopt the first interpretation in our discussion, we remain agnostic about the interpretation of gradient certainty ratings. 

At-issueness was measured with different diagnostics across Exps.~1-3. For polar questions (Exps.~1q, 2q, 3q), at-issueness was assessed with the three measures in (\ref{ai-question}):


\begin{exe}
\ex\label{ai-question} At-issueness diagnostics for polar questions
\begin{xlist}
\ex `asking whether $c$' 
\\ {\bf Christopher:} Did Cole discover that Julian dances salsa?
\\ Is Christopher asking whether Julian dances salsa?

\ex {\em yes}, $c$
\\ {\bf Christopher:} Did Cole discover that Julian dances salsa?
\\ {\bf Sandy:} Yes, Julian dances salsa.
\\ Does Sandy's response to Christoper sound good?

\ex {\em yes, but} $\neg c'$
\\ {\bf Christopher:} Did Cole discover that Julian dances salsa?
\\ {\bf Sandy:} Yes, but Cole didn't discover it. 
\\ Does Sandy's response to Christoper sound good?

\end{xlist}
\end{exe}

For declaratives, at-issueness was assessed with the three measures in (\ref{ai-decl}):

\begin{exe}
\ex\label{ai-decl} At-issueness diagnostics for declaratives 

\begin{xlist}

\ex `sure that $c$'
\\ {\bf Christopher:} Cole didn't discover that Julian dances salsa.
\\ \hspace*{2cm} Perhaps Cole discovered that Julian dances salsa.
\\ {\bf Sandy:} Are you sure?
\\ {\bf Christopher:} Yes, I'm sure that Julian dances salsa.
\\ Did Christopher answer Sandy's question?

\ex {\em yes, that's true}, $c$
\\ {\bf Christopher:} Cole didn't discover that Julian dances salsa.
\\ {\bf Sandy:} Yes, that's true, Julian dances salsa.
\\ Does Sandy's response to Christopher sound good?

\ex {\em yes, that's true, but} $\neg c'$
\\ {\bf Christopher:} Cole didn't discover that Julian dances salsa.
\\ {\bf Sandy:} Yes, that's true, but Cole discovered it. 
\\ Does Sandy's response to Christopher sound good?

\end{xlist}

\end{exe}

\begin{table}[h!]
\centering
\begin{tabular}{r | r | r | r}

Exp. & Embedding & Projection measure & At-issueness measure \\ 
\hline
1q & question & certain that $c$ & asking whether $c$ \\
1n & negation & certain that $c$ & sure that $c$ \\
1m & modal & certain that $c$ & sure that $c$ \\
1c & conditional & certain that $c$ & sure that $c$ \\
\hline
2q & question & certain that $c$ & {\em yes}, $c$ \\ 
2n & negation & certain that $c$& {\em yes, that's true}, $c$ \\ 
2m & modal & certain that $c$& {\em yes, that's true}, $c$ \\ 
2c & conditional & certain that $c$& {\em yes, that's true}, $c$ \\ 
\hline
3q & question & certain that $c$& {\em yes}, but $\neg c'$ \\ 
3n & negation & certain that $c$& {\em yes, that's true}, but $\neg c'$ \\ 
3m & modal & certain that $c$& {\em yes, that's true}, but $\neg c'$ \\ 
3c & conditional & certain that $c$& {\em yes, that's true}, but $\neg c'$ \\ 
\hline
\end{tabular}
\caption{Overview of the entailment-canceling embeddings, projection measures, and at-issueness measures in Exps.~1-3, where $c$ is the content under investigation, and $c'$ is a salient, different content.}\label{t-overview}
\end{table}

{\bf Comparisons for at-issueness:}

\begin{itemize}

\item Across Exps 1q, 2q, 3q (same assumption about at-issue content partitioning context set; reveals effect of embedding)

\item Across Exps n, m, c (same measure of at-issueness in Exps 1, 2, 3; reveals effect of embedding)

\item 2q -- 2n, 2m, 2c (yes...$c$; i.e., anaphor with $c$)

\item 3q -- 3n, 3m, 3c (yes...but $\neg c$, i.e., anaphor with $c$)


\end{itemize}

\footnote{Exps.~1-3 also allow us to address the question of whether the projection of contents embedded under different entailment-canceling operators is positively correlated, as assumed in the semantics/pragmatics literature (e.g., REFERENCES). Prior experimental literature: \citealt{smith-hall11,smith-hall-cls,demarneffe-etal-sub23}. As shown in Supplement \ref{a-projection}, this is the case.}

\subsection{Participants}

We recruited 250-300 participants for each of Exps.~1-3. Participants for Exp.~1q were recruited on Amazon's Mechanical Turk platform; the participants were required to have U.S.\ IP addresses and at least 99\% of previously approved HITs. Participants for the remaining experiments were recruited on Prolific; these participants were required to reside in the US, to be born in the US, to have English as their first language, and to have an approval rating of at least 99\%. Information on recruited participants (total number, ages, gender) and the payments they received can be found in Supplement \ref{a-participants}.

\subsection{Materials and procedure}

Exps.~1-3 investigated the projection and at-issueness of the CC of the 20 clause-embedding predicates in (\ref{pred}). The clausal complements of the 20 predicates were realized by 20 clauses (provided in Supplement \ref{a-clauses}), for a total of 400 predicate/clause combinations. These 400 predicate/clause combinations were combined with proper name subjects (a unique name per complement clause) to realize polar questions (in Exps. 1q, 2q, and 3q), negated declaratives (in Exps. 1n, 2n, and 3n), declaratives with the modal adverb {\em perhaps} (in Exps. 1m, 2m, and 3m), and the antecedents of conditionals (in Exps. 1c, 2c, and 3c). These sentences were presented as utterances by a named speaker. A sample set of target utterances is given in (\ref{sample-stims}): here, the named speaker is {\em Daniel}, the subject of the clause-embedding predicate is {\em Cole}, the predicate is {\em discover}, and the complement clause is {\em Julian dances salsa}.\footnote{The indirect object of {\em inform} was realized by the proper name {\em Sam}. The predicates differed in the tense in which they were realized. In polar interrogatives as well as in negated and modalized declaratives, eventive predicates, like {\em discover} and {\em hear}, were realized in the past tense and stative predicates, like {\em know} and {\em be annoyed}, were realized in the present tense.  In the antecedents of conditionals, all predicates were realized in the present tense. For the consequents of the conditional sentences see Supplement \ref{a-target}.}

\begin{exe}
\ex\label{sample-stims}
\begin{xlist}
\ex {\bf Daniel:} ``{\em Did Cole discover that Julian dances salsa?}"
\ex {\bf Daniel:} ``{\em Cole didn't discover that Julian dances salsa.}"
\ex {\bf Daniel:} ``{\em Perhaps Cole discovered that Julian dances salsa.}"
\ex {\bf Daniel:} ``{\em If Cole discovered that Julian dances salsa, Logan will be joyful.}"
\end{xlist}
\end{exe}
The proper names that realized the speakers, the subjects of the clause-embedding predicates, the subjects of the complement clauses, and the subjects of the consequents  were all unique.

Projection and at-issueness of the contents of the clausal complements were measured in separate blocks. In the projection blocks across Exps.~1-3, projection was measured with the `certain that' diagnostic. The target stimuli consisted of the target utterances, as shown in Figure \ref{f-projection-trials}.


\begin{figure}[h!]
\centering

\begin{subfigure}[t]{0.5\textwidth}
        \centering
\fbox{\includegraphics[height=4.4cm,width=7.6cm]{figures/2q-proj}}
\caption{Exps.~1q, 2q, and 3q.}\label{fig-exp1q-projection}
\end{subfigure}%
\begin{subfigure}[t]{0.5\textwidth}
\centering
\fbox{\includegraphics[height=4.4cm,width=7.6cm]{figures/1n-proj}} 
\caption{Exps.~1n, 2n, and 3n.}\label{fig-exp1n-projection}
 \end{subfigure}
\begin{subfigure}[t]{0.5\textwidth}
        \centering
\fbox{\includegraphics[height=4.4cm,width=7.6cm]{figures/1m-proj}}
\caption{Exps.~1m, 2m, and 3m}\label{fig-exp1m-projection}
 \end{subfigure}%
\begin{subfigure}[t]{0.5\textwidth}
\centering
\fbox{\includegraphics[height=4.4cm,width=7.6cm]{figures/1c-proj}} 
\caption{Exps.~1c, 2c, and 3c}\label{fig-exp1c-projection}
\end{subfigure}


\caption{Target trials in projection blocks of Exps.~1, 2, and 3 for the complement {\em Julian dances salsa}.}\label{f-projection-trials}
\end{figure}

The at-issueness diagnostics differed between Exps.~1, 2, and 3: at-issueness was measured in Exp.~1q, 2q, and 3q, with variants of the question-based diagnostic of at-issueness, namely the `asking whether' diagnostic in Exp.~1q, and the `yes, $p$' diagnostic in Exps.~2q and 3q, as shown in Figure \ref{fig-ai-trials-q}.


, with the `sure that' diagnostic in Exps.~1n, 1m, and 1c, with the `yes' diagnostic in Exps.~2q and 3q, with the `assent with positive continuation' diagnostic in Exps.~2n, 2m, and 2c,  and with the `assent with adversative continuation' diagnostic in Exps.~3n, 3m, and 3c.  


To assess whether participants were attending to the task, each experiment also included six control stimuli, which were also utterances made by a named speaker. (The controls are provided in Supplement \ref{a-control}.) Each participant saw a random set of 26 stimuli: each set contained one target utterance for each of the 20 clause-embedding predicates (each with a unique complement clause) and the same 6 control stimuli. Each participant saw their set of 26 stimuli twice, once in the projection block and once in the at-issueness block. Block order and within-block trial order was randomized. 


\begin{figure}[h!]
\centering

\begin{subfigure}[t]{0.5\textwidth}
        \centering
\fbox{\includegraphics[height=5.4cm,width=7.6cm]{figures/1q-ai}}
\caption{Exp.~1q.}\label{fig-exp1q-ai}
\end{subfigure}
\begin{subfigure}[t]{0.5\textwidth}
\centering
\fbox{\includegraphics[height=5.4cm,width=7.6cm]{figures/2q-ai}} 
\caption{Exp.~2q.}\label{fig-exp2q-ai}
 \end{subfigure}%
\begin{subfigure}[t]{0.5\textwidth}
        \centering
\fbox{\includegraphics[height=5.4cm,width=7.6cm]{figures/3q-ai}}
\caption{Exp.~3q}\label{fig-exp3q-ai}
 \end{subfigure}

\caption{Target trials in at-issueness blocks of Exps.~1q, 2q, and 3q for the complement {\em Julian dances salsa}.}\label{f-ai-trialsq}
\end{figure}

\begin{figure}[h!]
\centering

\begin{subfigure}[t]{0.5\textwidth}
        \centering
\fbox{\includegraphics[height=5.4cm,width=7.6cm]{figures/2n-ai}}
\caption{Exp.~2n.}\label{fig-exp1q-ai}
\end{subfigure}%
\begin{subfigure}[t]{0.5\textwidth}
\centering
\fbox{\includegraphics[height=5.4cm,width=7.6cm]{figures/2m-ai}} 
\caption{Exp.~2m.}\label{fig-exp2q-ai}
 \end{subfigure}
\begin{subfigure}[t]{0.5\textwidth}
        \centering
\fbox{\includegraphics[height=5.4cm,width=9cm]{figures/2c-ai}}
\caption{Exp.~2c}\label{fig-exp3q-ai}
 \end{subfigure}

\caption{Target trials in at-issueness blocks of Exps.~2n, 2m, and 2c for the complement {\em Julian dances salsa}.}\label{f-ai-trialsq}
\end{figure}


\begin{figure}[h!]
\centering

\begin{subfigure}[t]{0.5\textwidth}
        \centering
\fbox{\includegraphics[height=5.4cm,width=7.6cm]{figures/3n-ai}}
\caption{Exp.~3n.}\label{fig-exp1q-ai}
\end{subfigure}%
\begin{subfigure}[t]{0.5\textwidth}
\centering
\fbox{\includegraphics[height=5.4cm,width=7.6cm]{figures/3m-ai}} 
\caption{Exp.~3m.}\label{fig-exp2q-ai}
 \end{subfigure}
\begin{subfigure}[t]{0.5\textwidth}
        \centering
\fbox{\includegraphics[height=5.4cm,width=9cm]{figures/3c-ai}}
\caption{Exp.~3c}\label{fig-exp3q-ai}
 \end{subfigure}

\caption{Target trials in at-issueness blocks of Exps.~3n, 3m, and 3c for the complement {\em Julian dances salsa}.}\label{f-ai-trialsq}
\end{figure}


In the projection block, target stimuli consisted of a fact and a polar question that was ut- tered by a named speaker, as shown in Figure 1B. The polar questions were formed by real- izing the 20 clauses as the complements of the 20 clause-embedding predicates in Figure 1C. Participants were told to imagine that they are at a party and that, on walking into the kitchen, they overhear somebody ask somebody else a question. Projection was measured using the ?certain that? diagnostic (Dj�rv \& Bacovcin, 2017; Lorson, 2018; Mahler, 2020; Tonhauser et al., 2018): participants were asked to rate whether the speaker was certain of the CC, taking into consideration the fact. They gave their responses on a slider marked ?no? at one end (coded as 0) and ?yes? at the other (coded as 1). Greater speaker commitment to the CC should result in higher slider ratings.

Participants were told to imagine that they are at a party and that, on walking into the kitchen, they overhear somebody ask a question. Participants were asked to rate whether the speaker was certain of the CC. They gave their responses on a slider marked `no' at one end (coded as 0) and `yes' at the other (coded as 1), as shown in Figure \ref{fig-trial-exp1}.


After completing the experiment, participants filled out a short, optional survey about their age, their gender, their native language(s) and, if English is their native language, whether they are a speaker of American English (as opposed to, e.g.\ Australian or Indian English). To encourage truthful responses, participants were told that they would be paid no matter what answers they gave in the survey.

\subsection{Data exclusion} 

We excluded data from participants who took any experiment more than once and who did not self-declare to be native speakers of American English. We also excluded data from participants based on their ratings on the main clause controls and other criteria given in Supplement \ref{a-participants}. In each experiment, the data from between 215-281 remaining participants were analyzed. Information on the participants whose data entered into the analysis (total number, ages, gender) can be found in Supplement \ref{a-participants}.

\section{Experiments 1-3: Results}\label{s4}

\subsection{Comparing at-issueness}

\subsection{GPP}

Further analyses:

\begin{itemize}
\item look at how many unique participants

\item negative correlation between Q/A and assent might be compound of a) embedding (negation) and b) assent, because we don?t see negative correlation, but no correlation, with other embeddings (modal, conditional)

\item  compare variance for both projection and at-issueness, under the different embeddings and diagnostics: very little variance for assent diagnostic/negation embedding

\item have a look at the predicates that are exceptional to the GPP in Exps1 and 2, how do they behave across the other experiments? Exp3: looks like they are also their own little group

\item  when we compare negation, modal, conditional with assent diagnostic, we can see what effect embedding has

\item next experiments: no embedding of the 20 predicates, with assent diagnostic, to compare to prior literature who didn?t use embedding with the assent diagnostic; do assent diagnostic without ?that?s true? anaphor to engage with Snider?s assumption that ?yes? is not really anaphoric in the assent diagnostic and hence also not in the Q/A diagnostic

\end{itemize}

\section{General discussion}\label{s4}

\section{Conclusions}\label{s5}

% end document here for word count
%\end{document}

\bibliographystyle{cslipubs-natbib}
\bibliography{bibliography}

\newpage

\section*{Supplemental materials}

\appendix

\setcounter{page}{1}
%\renewcommand{\thetable}{A\arabic{table}}

\setcounter{table}{0}
\renewcommand{\thetable}{A\arabic{table}}

\setcounter{figure}{0}
\renewcommand{\thefigure}{A\arabic{figure}}

\section{20 complement clauses}\label{a-clauses}

The following clauses realized the complements of the predicates in Exps.~1-3:

\begin{enumerate}[leftmargin=3ex,itemsep=-2pt]

\begin{multicols}{2}

\item Mary is pregnant.
\item Josie went on vacation to France.
\item Emma studied on Saturday morning.
\item Olivia sleeps until noon.
\item Sophia got a tattoo.
\item Mia drank 2 cocktails last night.
\item Isabella ate a steak on Sunday.
\item  Emily bought a car yesterday.
\item  Grace visited her sister.
\item Zoe calculated the tip.

\columnbreak

\item  Danny ate the last cupcake.
\item  Frank got a cat.
\item  Jackson ran 10 miles.
\item  Jayden rented a car.
\item  Tony had a drink last night.
\item  Josh learned to ride a bike yesterday.
\item  Owen shoveled snow last winter.
\item  Julian dances salsa.
\item  Jon walks to work.
\item  Charley speaks Spanish.

\end{multicols}

\end{enumerate}

\section{Consequents for conditional target stimuli in Exps.~1c, 2c, and 3c}\label{a-target}

The consequents for the conditional target stimuli in Exps.~1c, 2c, and 3c were created with the following considerations in mind. Each complement clause was paired with a unique consequent clause, as shown in the list below for the 20 complement clauses. Each consequent clause consisted of a uniquely named subject and an adjectival predication in the future tense ({\em will be}); the adjectives all denoted an emotion. We selected the 20 emotion-denoting adjectives based on the following criteria: 10 of the adjectives had a positive valence, and 10 had a negative valence; all 20 adjective had an arousal value between 4.7 and 6.5 (based on the valence and arousal values reported in \citealt{warriner-etal2013}). 

\begin{enumerate}[leftmargin=3ex,itemsep=-2pt]

\item \ldots that Mary is pregnant, Esther will be mad.
\item \ldots that Josie went on vacation to France, Arnold will be frustrated.
\item \ldots that Emma studied on Saturday morning, Liam will be proud.
\item \ldots that Olivia sleeps until noon, Elijah will be embarrassed.
\item \ldots that Sophia got a tattoo, Ariel will be giddy.
\item \ldots that Mia drank 2 cocktails last night, Mariela will be worried.
\item \ldots that Isabella ate a steak on Sunday, Liz will be delighted.
\item  \ldots that Emily bought a car yesterday, Kate will be excited.
\item \ldots that  Grace visited her sister, Henry will be surprised.
\item \ldots that Zoe calculated the tip, Alex will be astonished.
\item  \ldots that Danny ate the last cupcake, Harper will be disgusted.
\item  \ldots that Frank got a cat, Lucas will be grouchy.
\item  \ldots that Jackson ran 10 miles, Kayla will be cheerful.
\item  \ldots that Jayden rented a car, Brittany will be furious.
\item  \ldots that Tony had a drink last night, Victoria will be ashamed.
\item  \ldots that Josh learned to ride a bike yesterday, Mason will be envious.
\item  \ldots that Owen shoveled snow last winter, Bianca will be jealous.
\item  \ldots that Julian dances salsa, Logan will be joyful.
\item  \ldots that Jon walks to work, Caleb will be suspicious.
\item  \ldots that Charley speaks Spanish, Jay will be happy.

\end{enumerate}


\section{Control stimuli in the projection blocks of Exps.~1-3}\label{a-control}

The control stimuli in the projection blocks of Exps.~1-3 were formed from the sentences in (1). In Exps.~1, we used the polar question variants of the sentences in (1) in Exp.~1q, and the (positive declarative) sentences in (1) in Exps.~1n, 1m, and 1c; in Exp.~1c, we added the non-restrictive relative clauses (NRRCs) provided in (1) to the respective subjects, so that the control stimuli have two clauses, like the target stimuli. 

\begin{exe}
\exi{(1)}  Sentences for control stimuli in Exps.~1
\begin{xlist}
\ex These muffins have blueberries in them.  (NRRC: , which are really delicious, )
\ex This pizza has mushrooms on it. (NRRC: , which I just made from scratch, )
\ex Jack was playing outside with the kids. (NRRC: , who is my long-time neighbor, )
\ex Ann dances ballet. (NRRC: , who is a local performer, )
\ex John's kids were in the garage. (NRRC: , who are very well-behaved, )
\ex Samantha has a new hat. (NRRC: , who is really into fashion, )
\end{xlist}
\end{exe}
We expected participants to give low responses in the `certain that'  diagnostic  in Exp.~1q (indicating that the speaker is not certain of the main clause content, i.e., that the content does not project out of the question) and high responses in the `certain that'  diagnostic in Exps.~1n, 1m, and 1c (indicating that the speaker is certain of the main clause content). These expectations were borne out, as shown in the first three rows of Table \ref{t-proj-controls}.

%We used positive declarative control sentences in Exps.~1n, 1m, and 1c so that we could have clear expectations about participants' responses. Thus, we expected participants to give low responses in the `certain that' and `asking whether' diagnostic in Exp.~1q (indicating that the main clause content does not project and is at-issue), and we expected participants to give low responses in the `certain that' diagnostic and high responses in the `sure that' diagnostic in Exps.~1n, 1m, and 1c (indicating, again, that the main clause content does not project and is at-issue). 


In Exps.~2 and 3, the NRRCs were included in all control stimuli: the carrier sentences were the polar question variants of the sentences in (1) in Exps.~2q and 3q, and the sentences in (1) for the remaining experiments. The NRRCs were included in Exps.~2 and 3 to make the use of the assent diagnostics in the at-issueness blocks more natural: like the target stimuli, the control stimuli consist of two clauses, one of which the relevant speaker assents with.

We had the same expectations as for the control stimuli in Exps.~1: low responses in the `certain that' diagnostic in Exps.~2q and 3q (indicating that the speaker is not certain of the main clause content, i.e., that the content does not project out of the question), and high responses in the `certain that' diagnostic in the remaining experiments (indicating, again, that the speaker is certain of the main clause content). These expectations were borne out, as shown in the lower nine rows of Table \ref{t-proj-controls}.


\begin{table}[h!]
\centering
\begin{tabular}{r r}
Experiment & mean certainty rating \\ 
\hline
1q & .14 \\
2q & .18 \\
3q & .17 \\
\hline
1n & .95 \\
1m & .96 \\
1c & .94 \\
\hline
2n & .96 \\
2m & .96 \\
2c & .96 \\
\hline
3n & .94 \\
3m & .93 \\
3c & .93 \\
\hline
\end{tabular}
\caption{Mean certainty ratings for control stimuli, for self-declared American English participants}\label{t-proj-controls}
\end{table}

\newpage

\section{Ratings for main clause contents in at-issueness blocks in Exps.~1-3}\label{a-control-ai}

In the at-issueness blocks of Exps.~1-3, we collected at-issueness ratings for the main clause contents of the control stimuli of the projection blocks described in Supplement \ref{a-control}. Originally, the intent was to exclude participants' data on the basis of their at-issueness ratings, in parallel to their certainty ratings. We expected the main clause content to be at-issue across the diagnostics, which means that we expected low responses across all at-issueness diagnostics (recall that participants' responses were coded so that the higher a participants' response, the more not-at-issue the content was hypothesized to be, to investigate whether there is a positive correlation between projection and not-at-issueness). As shown in Table \ref{t-ai-controls}, these expectations were not always borne out, which is why we decided to not use participants' ratings of the main clause contents in the at-issueness blocks as an exclusion criterion.

\begin{table}[h!]
\centering
\begin{tabular}{r l}
Experiment & mean rating \\ 
\hline
1q & .05 \\
2q & .07 \\
3q & .28 \\
\hline
1n & .04 \\
1m & .03 \\
1c & .08 \\
\hline
2n & .22 (16 participants excluded by specified exclusion criterion) \\
2m & .25 \\
2c & .22 \\
\hline
3n & .44  (nobody excluded by specified exclusion criterion) \\
3m & .5 \\
3c & .53 \\
\hline
\end{tabular}
\caption{Mean ratings on at-issueness diagnostics for main clause contents in Exps.~1-3, for self-declared American English participants}\label{t-ai-controls}
\end{table}

\section{Data exclusion}\label{a-participants}

This supplement provides information on the recruited participants, the criteria by which participants' data were excluded, the remaining participants (that is, the participants whose data entered into the analysis), and the payment for each of Exps.~1-3. Table \ref{f-exclusion} provides information on the recruited and remaining participants, including the total number, the range of the participants' ages, the mean ages of the participants, and their self-reported gender (`f' = female, `m' = male, `o' = other, `u' = undeclared). No gender data was collected in Exp.~1q. Participants' data were excluded based on the following criteria:

\begin{itemize}[itemsep=-2pt]

\item `multiple': Due to an experimental glitch, some participants participated more than once. Since no information was available on which one was their first take, those participants' data was removed (which means that for each participant who took the experiment twice, at least two data sets were removed).

\item `language': Participants' data were excluded if they did not self-identify as native speakers of American English.

\item `controls': Participants' data were excluded if their mean rating on the 6 main clause control items in the projection block was more than 2 sd above the group mean (in Exps.~1q, 2q, and 3q) or more than 2 sd below the group mean (in the remaining experiments). Participants' data were also excluded if their mean rating on the 6 main clause control items in the at-issueness block was more than 2 sd above the group mean (across Exps.~1-3).
 
\item `variance': Participants' data were excluded if they always selected roughly the same point on the response scale for the target stimuli. We identified such participants by identifying participants whose mean variance on the target stimuli was more than 2sd below the group mean variance on the target stimuli, and manually inspecting their response patterns. %Exps 1q, 2q, 3q: participants with low variance not excluded after manual inspection
\end{itemize}

Participants took around 9-11 minutes to complete the various experiments. Participants were paid more in Exps.~1c, 2c, and 3c than the remaining experiments because the target stimuli in those experiments were longer (as they consisted of conditionals). More women than men were recruited in many of the experiments because they were run at a time when Prolific apparently went viral on TikTok, resulting in a large number of young women registering for the service (around July 24, 2021; see \url{https://blog.prolific.co/we-recently-went-viral-on-tiktok-heres-what-we-learned/}, \\ last accessed February 4, 2022).  

\begin{sidewaystable}[h!]
\centering
\begin{tabular}{l | r r r | r r r r | r r r | r }
&  \multicolumn{3}{c|}{Recruited participants} & \multicolumn{4}{c|}{Exclusion criteria} & \multicolumn{3}{c|}{Remaining participants} &  \\ 
Exp. & total & ages (mean) & f/m/o/u & multiple & language & controls & variance & total & ages (mean) & f/m/o/u & payment   \\ 
\hline
1q & 300  & 19-74 (38.2)  & --  & 5  & 7  & 35  & 0  & 242   & 21-74 (39.2) &  -- & \\
1n & 300  & 18-74 (33.2)  &  150/145/5/0 & 0  & 8 & 17 & 1 & 274  & 18-74 (33.3) & 141/128/5/0 & \$1.70\\
1m & 300  & 18-74 (32.7) & 150/141/7/2  &  0 & 0  & 19 & 0   & 281   & 18-74 (32.7) & 144/129/7/1 & \$1.70 \\
1c &  300 & 18-58 (25.9)  & 249/45/6/0 & 0 & 6  & 26 & 2  & 266  &  18-58 (24.8) & 235/25/6/0  & \$1.70 \\
2q & 250  &  18-58 (25.5) &  201/43/6/0  & 0 & 4  & 24  & 1 & 220  & 18-58 (24.8)  & 187/28/5/0  & \$2.15 \\
2n & 250  &  18-69 (33.2)  &  127/114/6/1 & 1  & 4  & 29 & 0& 215  &  18-69 (33.1) & 113/95/6/1    & \$1.70\\
2m & 251  & 18-74 (31.7)  & 132/113/6/0  & 0  & 4  & 27 &  0 & 220  & 18-70 (31.9) & 116/98/6/0 & \$1.70\\
2c &  250 &  18-56 (24.5)  & 212/30/8/0  & 0  & 0  & 26 & 0  & 224  & 18-56 (24.4) & 195/24/5/0 & \$2.15\\
3q & 250  &  18-66 (32.4) &  140/102/7/1 &  0 & 4  & 20  & 0  &  225 & 18-66 (32.6) & 125/93//7/0  & \$1.70 \\
3n & 250  &  18-70 (24.6) & 114/31/5/0  & 0  & 5  &  13 & 4  & 228  & 18-70 (24.3) &  198/25/5/0 &\$1.70 \\
3m & 250  & 18-63 (25.5)  & 205/40/5/0 &  0 & 3  & 14 &  0 & 233  &  18-63 (24.8) & 197/31/5/0 &\$1.70 \\
3c & 250  & 18-59  (27.5) & 182/64/4/0  & 0  & 3  & 17 & 0 & 230  &  18-59 (26.7)  & 177/49/4/0  & \$2.15\\
\end{tabular}
\caption{Recruited participants, excluded data, and remaining participants in Exps.~1, 2 and 3}\label{f-exclusion}
\end{sidewaystable} 

\section{Comparing projection with different entailment-canceling environments}\label{a-projection}
 
\end{document}

